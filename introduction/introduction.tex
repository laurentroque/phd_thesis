\pagenumbering{arabic}

\chapter{Introduction}

\ifpdf
    \graphicspath{{introduction/figs/Raster/}{introduction/figs/PDF/}{introduction/figs/}}
\else
    \graphicspath{{introduction/figs/Vector/}{introduction/figs/}}
\fi


%\section[Short title]{Reasonably long section title}



% =========================================
\section{The EDGES experiment}\label{sec:edges}
To date, the most significant result in 21-cm experimentation has been made by the Experiment to Detect the Global EoR Signature (EDGES) which aims to detect the sky-averaged 21-cm brightness temperature from the EoR \citep{edgesCal}. The project has been conducting multiple observations from the Murchison Radio-astronomy Observatory in Western Australia since 2006 \citep{edgesCal} using multiple dipole-like antennas of metal panels mounted horizontally above a ground plane \citep{edgesNature}. Early measurements placed an upper limit on the relative brightness temperature of the redshifted 21-cm signal contribution to their recorded foreground-removed spectrum \citep{edges2008} as well as a lower limit to the duration of the reionisation epoch with $\delta z > 0.06$, the latter result effectively excluding rapid reionisation models \citep{edges2010}. Following the deployment of one high-band and two low-band instruments, EDGES reported the detection of a flattened absorption profile in the radio spectrum centred at 78 MHz with a width of 19 MHz and depth of 0.5 K which they suggest is the 21-cm hydrogen signature \citep{edgesNature}.

The finding was met with considerable discussion as the detected profile’s characteristics did not match theoretical models. The trough centering at 78 MHz (corresponding to a redshift $z \sim 18$) would require more efficient star and galaxy formation at high redshifts \citep{edges_star_formation} while its flattened Gaussian shape suggest a delayed start to X-ray heating after the formation of Lyman-alpha emitting stars, not consistent with models \citep{theory_models}. Most notably however, was the profile amplitude which is more than a factor of two greater than the largest predictions by \citet{theory_models} which would indicate that either primordial gas was cooler than expected or that the background radiation temperature was hotter than expected \citep{edgesNature}. With both the radiation and gas temperatures constrained by the CMB and adiabatic cooling mechanisms, known astrophysical processes are unlikely to account for the observed discrepancy and new physics have been proposed to to rectify the inconsistency such as an IGM cooling channel facilitated by dark matter-baryon interactions \citep{edgesNature}. Other phenomena such as an excess radio background due to efficient black hole formation obscured by dense hydrogen halos have also been proposed \citep{ew_radio_background}.

Alternatively, inaccurate analysis methods or instrumental systematics may also account for the disparity between the EDGES data and theoretical models. \citet{hills_concerns} showed that the EDGES modelling process implied unphysical foreground emission parameters while \citet{sims_concerns} describe systematic calibration errors preferred by the Bayesian evidence under statistical analyses of the publicly available EDGES data\footnote{available at \url{http://loco.lab.asu.edu/edges/edges-data-release/}}. The SARAS 3 radiometer measuring radio sky spectra at 55--85 MHz tested for the presence of the EDGES best-fit profile which was rejected from their data at the 95.3\% confidence level \citep{saras_reject}. Furthermore upper limits on the 21-cm power spectrum set by HERA Phase I observations were found to neither support nor disfavour a cosmological origin to the feature seen by EDGES \citep{hera_limits}.

The divided interpretation of the profile centred at 78 MHz highlights the need for follow-up experimentation to definitively confirm or refute the findings of \citet{edgesNature}. Continued investigations will need to improve on the instrumentation, analysis methods and measurement techniques in order to avoid data of nebulous origin or questionable interpretation. Many projects such as Probing Radio Intensity at high-Z from Marion (PRIZM) \citep{prizm}, Mapper of the IGM Spin Temperature (MIST) \citep{mist} and the Dark Ages Polarimeter PathfindER (DAPPER) \citep{dapper} are spearheading the effort including a certain Cambridge-led collaboration…


% =========================================
\section{The REACH experiment}\label{sec:reach}
The Radio Experiment for the Analysis of Cosmic Hydrogen (REACH) is a Cambridge-led sky-averaged 21-cm experiment of which the author is a member of. The radiometer targeting the cosmic hydrogen signature between 50--170 MHz ($z \sim $7.5--28) has a principle science objective of verifying the EDGES detection through a phased deployment at the RFI-quiet Karoo South African radio reserve \citep{reach}. A design focus on the identification and removal of residual systematics sets REACH apart from concurrent experiments which typically aim to minimise the chromatic response of the instrument as achromatic antennas mitigate the introduction of spectral components into foreground signals \citep{edges2008,saras2}. Relying on such a ‘smooth’ instrument response presents challenges such as with EDGES where the resulting frequency bandwidth ratio is restricted to 2:1 necessitating the use of scaled systems with overlapping bands to cover the full frequency range. REACH alternatively intends to improve on current experiments through the joint detection and characterisation of instrument systematics together with astrophysical foregrounds and the cosmic signature using Bayesian statistical methods \citep{dom,dom_modelling}. This joint fit between physically-based models is expected to facilitate the detection of correlated systematics and avoid those potentially degenerate with a cosmological signal with unaccounted-for systematics diagnosed using Maximally Smooth Functions \citep{maxsmooth}.

REACH Phase I plans on taking simultaneous observations of the sky using two antennas; the hexagonal dipole (50--130 MHz) and the conical log spiral (50--170 MHz) both of which were chosen from numerous designs for their ability to reconstruct mock 21-cm signals to a high degree of statistical confidence with small root mean square error in simulations \citep{dom_antenna}\footnote{We note that the hexagonal dipole is similar to the rectangular dipole antenna used in the EDGES experiment}. The antenna pair will be analysed in parallel to isolate signal components associated with hardware systematics while the contrasting mechanical designs will prevent experimental sensitivity to hardware-specific systematics. The project’s non-reliance on achromatic antennas allows for an ultra-wideband system covering the Cosmic Dawn and EoR (up to 3.5:1 for the conical log spiral) \citep{john_antenna}. Using such a large bandwidth, spectral differences between the oscillating 21-cm signature and the smooth power-law foregrounds can be leveraged during data analysis to support a positive detection \citep{reach}.

The deployment site was also chosen with care after an extensive survey. In the Karoo, REACH will observe the southern hemisphere sky from a remote 4 km wide basin surrounded by hills and mesas offering a reduced FM radio presence. Located near similar experiments such as HERA, MeerKAT \citep{meerkat} and the Square Kilometre Array (SKA)1-Mid instrument \citep{ska}, the location offers critical support infrastructure including on-site maintenance, staff as well as controlled access through paved roads. A Phase II experiment is planned for the same site which will incorporate additional antenna systems. Scaled versions of the hexagonal dipole and dual polarisation antennas have been proposed.

Simulations run with the above specifications forecast percent-level constraints on astrophysical parameters using REACH and in the case of a non-detection, upper limits on the strength of the absorption feature can be used to bound high-redshift phenomena \citep{reach}. We estimate a $\sim 500$ mK absorption profile consistent with EDGES can be detectable in as little as three hours of integrated data out of approximately 30 hours of observation time ($\lesssim \frac{1}{6}$th the time necessary to detect more conservative 21-cm signature models at the same redshift), though up to an order of magnitude more time may need to be allocated for the removal of low quality data and calibration measurements. If the data do not support an EDGES-like or high-amplitude 21-cm signal, the experiment will continue to integrate for longer periods in which case REACH will be able to place rigorous constraints on any excess radio background amplitudes and hydrogen cooling mechanisms beyond the typical adiabatic expansion. For the case of a detected 21-cm absorption signature much smaller than the EDGES profile in the high signal-to-noise regime, forward modelling consistent with standard astrophysics and cosmology will be used to constrain parameters such as the low-energy cutoff frequency and power law index of the X-ray spectral energy distribution, the X-ray efficiency of sources, the CMB Thomson-scattering optical depth, the minimum virial circular velocity of star-forming galaxies, the mean free path of ionising photons in the IGM as well as the star-formation efficiency \citep{tom_crh,visbal,fialkov_history}.

Contingent on this detection is of course, the measurement accuracy and sensitivity of the instrument as a whole. The contribution to this project by the author is the maximisation of such attributes through the construction of a high quality radio-frequency receiver unit, its seamless incorporation within the overarching system, as well as the development of intelligent software for receiver calibration. In this thesis, we present the designs, motivation, specifications and production of the REACH front-end and back-end receiver unit for deployment in 2023 highlighting its unique architecture and features to aid in instrument characterisation. We also detail an innovative calibration procedure that incorporates a Bayesian methodology leading to an in-field, rapid algorithm for calibration of the instrument in the same environment as observations, a first of its kind. While presenting the results of the technique, we highlight the challenges of achieving the accuracy needed to detect such miniscule 21-cm signals as well as present a framework for continued advancements in calibration towards the detection of the highly-redshifted global 21-cm hydrogen signature.

