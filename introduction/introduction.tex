\pagenumbering{arabic}

\chapter{Introduction}

\ifpdf
    \graphicspath{{introduction/figs/Raster/}{introduction/figs/PDF/}{introduction/figs/}}
\else
    \graphicspath{{introduction/figs/Vector/}{introduction/figs/}}
\fi


%\section[Short title]{Reasonably long section title}



% =========================================
\section{The EDGES experiment}\label{sec:edges}
To date, the most significant result in 21-cm experimentation has been made by the Experiment to Detect the Global EoR Signature (EDGES) which aims to detect the sky-averaged 21-cm brightness temperature from the EoR \citep{edgesCal}. The project has been conducting multiple observations from the Murchison Radio-astronomy Observatory in Western Australia since 2006 \citep{edgesCal} using multiple dipole-like antennas of metal panels mounted horizontally above a ground plane \citep{edgesNature}. Early measurements placed an upper limit on the relative brightness temperature of the redshifted 21-cm signal contribution to their recorded foreground-removed spectrum \citep{edges2008} as well as a lower limit to the duration of the reionisation epoch with $\delta z > 0.06$, the latter result effectively excluding rapid reionisation models \citep{edges2010}. Following the deployment of one high-band and two low-band instruments, EDGES reported the detection of a flattened absorption profile in the radio spectrum centred at 78 MHz with a width of 19 MHz and depth of 0.5 K which they suggest is the 21-cm hydrogen signature \citep{edgesNature}.

The finding was met with considerable discussion as the detected profile’s characteristics did not match theoretical models. The trough centering at 78 MHz (corresponding to a redshift $z \sim 18$) would require more efficient star and galaxy formation at high redshifts \citep{edges_star_formation} while its flattened Gaussian shape suggest a delayed start to X-ray heating after the formation of Lyman-alpha emitting stars, not consistent with models \citep{theory_models}. Most notably however, was the profile amplitude which is more than a factor of two greater than the largest predictions by \citet{theory_models} which would indicate that either primordial gas was cooler than expected or that the background radiation temperature was hotter than expected \citep{edgesNature}. With both the radiation and gas temperatures constrained by the CMB and adiabatic cooling mechanisms, known astrophysical processes are unlikely to account for the observed discrepancy and new physics have been proposed to to rectify the inconsistency such as an IGM cooling channel facilitated by dark matter-baryon interactions \citep{edgesNature}. Other phenomena such as an excess radio background due to efficient black hole formation obscured by dense hydrogen halos have also been proposed \citep{ew_radio_background}.

Alternatively, inaccurate analysis methods or instrumental systematics may also account for the disparity between the EDGES data and theoretical models. \citet{hills_concerns} showed that the EDGES modelling process implied unphysical foreground emission parameters while \citet{sims_concerns} describe systematic calibration errors preferred by the Bayesian evidence under statistical analyses of the publicly available EDGES data\footnote{available at \url{http://loco.lab.asu.edu/edges/edges-data-release/}}. The SARAS 3 radiometer measuring radio sky spectra at 55- -85 MHz tested for the presence of the EDGES best-fit profile which was rejected from their data at the 95.3\% confidence level \citep{saras_reject}. Furthermore upper limits on the 21-cm power spectrum set by HERA Phase I observations were found to neither support nor disfavour a cosmological origin to the feature seen by EDGES \citep{hera_limits}.

The divided interpretation of the profile centred at 78 MHz highlights the need for follow-up experimentation to definitively confirm or refute the findings of \citet{edgesNature}. Continued investigations will need to improve on the instrumentation, analysis methods and measurement techniques in order to avoid data of nebulous origin or questionable interpretation. Many projects such as Probing Radio Intensity at high-Z from Marion (PRIZM) \citep{prizm}, Mapper of the IGM Spin Temperature (MIST) \citep{mist} and the Dark Ages Polarimeter PathfindER (DAPPER) \citep{dapper} are spearheading the effort including a certain Cambridge-led collaboration…


