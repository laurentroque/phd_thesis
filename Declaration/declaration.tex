% ******************************* Thesis Declaration ***************************

\begin{declaration}

This thesis is the result of my own work and includes nothing which is the outcome of work done in collaboration except as declared below and specified in the text. It is not substantially the same as any work that has already been submitted before for any degree or other qualification except as declared in the preface and specified in the text. It does not exceed the prescribed word limit for the Faculty of Physics \& Chemistry Degree Committee.

\Cref{sec:21cm} reviews decades of research in 21-cm cosmology following the work of many authors cited throughout. \Cref{sec:edges} summarises the work done in relation to the experiments published as \citet{edgesNature} with \cref{sec:historic_cal} detailing the calibration methods used in the EDGES experiment introduced by \citet{rogersCal} and \citet{edgesCal}, though some equations have been changed in this thesis to conform with modern notation. \Cref{sec:receiver_general}, \cref{sec:additional_models} and \cref{sec:matlab_results} are intended to be published as \mbox{\citet{nimaCal}} which was initially co-written with Dr. Nima Razavi-Ghods. The work presented in this thesis has been rewritten and expanded to incorporate details not included in \citet{nimaCal}. Computer aided design (CAD) models for various devices developed in this work were created by Steven H. Carey and are credited as such in the image descriptions. Some of the CAD images and files have been altered by the author for presentation in this thesis. The experiment regarding the front-end thermal management system effectiveness on six litres of water was done independently by Steven H. Carey but is included here for completeness. \Cref{sec:reach_formalism}, \cref{sec:likelihood} and \cref{sec:simulated_data} have been published as \citet{ian_bayes}. \Cref{sec:mphil_results} summarises work submitted to the University of Cambridge for an MPhil degree entitled ‘Bayesian Techniques for the Calibration of 21 cm Global Experiments’. \Cref{sec:fbf} is work intended to be published by the author at a later date. \Cref{fig:amp1_schematic}, \cref{fig:amp2_schematic} and \cref{fig:fem_schematic} are circuit diagrams created by John A. Ely based on designs by Dr. Nima Razavi-Ghods.


\end{declaration}
