% ************************** Thesis Abstract *****************************
% Use `abstract' as an option in the document class to print only the titlepage and the abstract.
\begin{abstract}
The detection of minute radio-frequency signals from the primordial Universe are thought to contain fundamental information on the evolution of the first luminous sources. Such breakthroughs however are hindered by the unprecedented levels of sensitivity and calibration needed to confidently distinguish these millikelvin-level signatures from galactic foregrounds and instrument systematics. In this work we detail the development of a calibration methodology that expands upon the Dicke switching procedure introduced for microwave-frequency devices and apply it to contemporary experiments targeting early time periods such as the Dark Ages, Cosmic Dawn and Epoch of Reionisation.

Included are the designs and practical considerations for a receiver unit housing numerous calibration standards, a compact microcontroller unit, portable vector network analyser and Peltier-based thermal management system for deployment with the REACH radiometer experiment in the South African Radio Astronomy Observatory. Following this, we detail a first-of-its-kind Bayesian calibration algorithm named \textsc{Excalibrate} which offers unparalleled speed and mobility, allowing for the characterisation of the radiometer in the same environment as observational measurements. Datasets taken at various points of the receiver development are evaluated with \textsc{Excalibrate} which achieves calibration accuracies of about 1 kelvin or less.

Upon numerous adjustments to both the physical receiver unit and our code, we demonstrate that the polynomial approximation for calibration parameters used by \textsc{Excalibrate} may not be an appropriate model for continued advancement towards a calibration accuracy less than ten millikelvin. In light of this, we derive a mathematical framework for an alternative method to solve for calibration parameters as singular values at each frequency point and conclude with further suggestions for increasing the sensitivity of the radiometer.

\end{abstract}
